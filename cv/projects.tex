%-------------------------------------------------------------------------------
%	SECTION TITLE
%-------------------------------------------------------------------------------
\cvsection{Highlighted Software Projects}

%-------------------------------------------------------------------------------
%	CONTENT
%-------------------------------------------------------------------------------
\begin{cventries}

  \cventry
    {Delta Solutions} % Organisation
    {AhLingo} % Project
    {Online} % Location
    {2023} % Date(s)
    {
      \begin{cvitems}
        \item {Innovative platform harnessing LLMs and AI to facilitate language learning for users.}
        \item {Integrates advanced algorithms to provide intuitive and adaptive language lessons.}
        \item {\url{https://github.com/aoifehughes/ahlingo}}
      \end{cvitems}
    }

    \cventry
    {Delta Solutions} % Organisation
    {PoliwhiRL} % Project
    {Online} % Location
    {2024} % Date(s)
    {
      \begin{cvitems}
        \item {Machine learning library using reinforcement learning to optimise
        pathing in comptuer games}
        \item {Uses computer vision and Deep-Q-Networks to train agents to navigate complex environments}
        \item {\url{https://github.com/aoifehughes/PoliwhiRL}}
      \end{cvitems}
    }

  \cventry
    {The Alan Turing Institute} % Organisation
    {Ethical Assurance Platform} % Project
    {England, UK} % Location
    {2023} % Date(s)
    {
      \begin{cvitems}
        \item {Platform designed to evaluate the ethical implications of AI systems.}
        \item {Employs Django and React frameworks for seamless assurance case creations by non-technical users.}
        \item {\url{https://github.com/alan-turing-institute/AssurancePlatform}}
      \end{cvitems}
    }

  \cventry
    {The Alan Turing Institute} % Organisation
    {Pandemia} % Project
    {England, UK} % Location
    {2022} % Date(s)
    {
      \begin{cvitems}
        \item {Framework for pandemic modeling using C and Python.}
        \item {Collaborated with Imperial University researchers to bolster software engineering practices, emphasizing openness and adaptability.}
        \item {\url{https://github.com/PandemiaProject/pandemia}}
      \end{cvitems}
    }

  \cventry
    {John Innes Centre} % Organisation
    {Narrow Escape Simulator} % Project
    {England, UK} % Location
    {2021} % Date(s)
    {
      \begin{cvitems}
        \item {Developed during Ph.D, this stochastic simulation offers numerical solutions to the narrow escape problem using efficient coding.}
        \item {\url{https://github.com/AoifeHughes/PyEscape}}
      \end{cvitems}
    }

  \cventry
    {The Alan Turing Institute} % Organisation
    {Data Science Course} % Project
    {England, UK} % Location
    {2022} % Date(s)
    {
      \begin{cvitems}
        \item {Open source initiative offering structured coursework and self-study resources for research data science.}
        \item {Contributed in collaboration with the Turing research engineering team.}
        \item {\url{https://github.com/alan-turing-institute/rds-course}}
      \end{cvitems}
    }

  \cventry
    {National Plant Phenomics Centre} % Organisation
    {X-ray Analysis Software for Wheat Grains} % Project
    {Wales, UK} % Location
    {2018} % Date(s)
    {
      \begin{cvitems}
        \item {Open-source toolkit to automate morphometric data extraction from 3-D x-ray plant imagery.}
        \item {\url{https://github.com/NPPC-UK/microCT_grain_analyser}}
      \end{cvitems}
    }

\end{cventries}
